\label{ch:updates} 



\section{Updates}

\subsection*{\indent Week 7 - 12/11/18}


\textbf{Tasks completed:}

The collection of data is complete with a total of 900 days (08/10/2018 - 22/04/2016) of weather data for each of the 8 locations. I've put into place the framework to clean the data e.g removing redundant data, ensuring missing data is dealt with etc...

The data comes with a total of 33 different features of varying use. I've started to investigate the removal of correlated features. Since I have a lot of different features, I need to determine the effect multicollinearity (one feature can be  produced relatively accurately by a combination of others) and negate it if present. This could be by dropping or combining features together

I've started to implement simple Gaussian process regression between two different features. I still need to read more around how to do a multiple input Gaussian regression (e.g using 5 different features' values to predict just a single one). 


\textbf{Problems and challenges:}


Ensuring this project is computer science project rather than statistics -  do I need to make sure I'm taking a Bayesian rather than Frequentist approach?

\textbf{Task for the coming week}


Two different avenues to explore: fuzzy Bayesian Network or a spatio-temporal kernel.

Potentially a convolution or combination of a spatial kernel and a termporal kernel 


Read more papers about using a spatio-temporal kernel in the Gaussian process regression and look at how it is implemented.



Start to write the introduction 