






\section{Paper Reviews}


\subsection{Paper 1}

Spatiotemporal Prediction of Ambulance Demand using Gaussian Process Regression

https://arxiv.org/pdf/1806.10873.pdf




\subsection{Paper 2}
“Estimation and prediction of weather variables from surveillance data using spatio-temporal Kriging”\\ \url{https://upcommons.upc.edu/bitstream/handle/2117/112695/dalmau_17_kriging.pdf}

Summary of report:


\begin{itemize}


\item{}AIR TRAFFIC CONTROL
\item{}NWP = numerical weather predictions
\item{}"Kriging is a geostatistical interpolation technique to create short- term weather predictions from scattered weather observations “
\item{}Most interpolation methods estimate variable as weighted sum of observations from location  further away = lower weight
\item{}Geostatistical – data-driven statistical models that consider correlation between data e.g Kriging
\item{}Kriging interpolation provides best estimate of variable Z at unmeasured location x from set of surrounding data points


\item{}Two different methods : 
\begin{itemize}
    


\item{}do temporal regression for all cities then spatio regression to take into account the links between cities. i.e predict value for all cities, then use those values in kriging to predict a value for a single city.



\end{itemize}


\item{}Two types of spatio-temporal variogram:

\begin{itemize}


 
\item{}separable = combination of purely spatial and purely temporal variograms 
\item{}non-separable. =. “more flexible to handle…” ( above equation 17)
spatio-temporal UK (UK-ST) variant 

\end{itemize}
\end{itemize}

\subsection{Paper 3}
Improved space–time forecasting of next day ozone concentrations in the eastern US 

\url{http://www.soton.ac.uk/~sks/research/papers/sahuyipholland.pdf}


 

https://www.sciencedirect.com/science/article/pii/S2211675313000195

Spatio-temporal modeling for real-time ozone forecasting



Sahu, S.K., Yip, S., Holland, D.M., 2009a. A fast Bayesian method for updating and forecasting hourly ozone levels. Environmental 
and Ecological Statistics 18, 185–207.
Sahu, S.K., Yip, S., Holland, D.M., 2009b. Improved space–time forecasting of next day ozone concentrations in the eastern US. 
Atmospheric Environment 43, 494–501.




\subsection{Paper 4}

"A Bayesian spatio-temporal model for forecasting Anaplasma species seroprevalence in domestic dogs within the contiguous United States"

\url{https://journals.plos.org/plosone/article/file?id=10.1371/journal.pone.0182028&type=printable}

\begin{itemize}




    \item Looking at ticks in dogs across USA
    \item Show correlation between features to backup point of spatial correlation
    \item They use bayesain hierarchical spatio-temporal regression model, autocorrelated random effects are utilized to account for the spatio and temporal dependence.
    \item areal units =  places ----> use this term
    \item $Y_s(t)$ is positive test for county s at year t
    \item they have a term for "spatio-temporal random effects used to account for the spatial and temporal dependence"
    \item conditional autoregressive model (CAR) captures the spatial dependence
    \item "markov chain monte carlo posterior sampling algorithm"
    
    
    
    
    
\end{itemize}


\subsection{Paper 5}

A Bayesian hierarchical spatio-temporal model for extreme rainfall in Extremadura (Spain)





\subsection{Paper 6}


Bowman DD, Liu Y, McMahan CS, Nordone SK, Yabsley MJ, Lund RB. Forecasting United States heartworm Dirofilaria immitis prevalence in dogs. Parasit Vectors. 2016; 9(1):540. 

\url{https://doi.org/10. 1186/s13071-016-1804-y}


\subsection{Paper 7}

"A Probabilistic Approach for Weather Forecast using Spatio-temporal Inter-relationships among Climate Variables"
